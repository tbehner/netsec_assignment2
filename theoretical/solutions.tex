\documentclass{scrartcl}
\usepackage[ngerman]{babel}
\usepackage[utf8]{inputenc}
\usepackage[T1]{fontenc}
\usepackage{lmodern}
\usepackage[dvipsnames,svgnames]{xcolor}
\usepackage{graphicx}
\usepackage{enumerate}
\usepackage{multicol}
\usepackage{float}
\usepackage{hyperref}
\definecolor{darkblue}{rgb}{0,0,.5}
\hypersetup{pdftex=true, colorlinks=true, breaklinks=true, linkcolor=darkblue, menucolor=darkblue, pagecolor=darkblue, urlcolor=darkblue}

\usepackage{amsmath, amssymb, stmaryrd}
\usepackage{amsthm, mathtools}

\theoremstyle{definition}
\newtheorem{exercise}{Task}

\usepackage{tikz}
\title{Assignment Sheet 2}
\author{Timm Behner \and Christopher Kannen \and Eva-Lotta Teutrine}
\date{\today}

\begin{document}
\maketitle
\begin{exercise}\hfill
    For Information about hellgate see task\_2.1.txt. For information about
    other computer found on the network see netscan.log.
    Candy:
    \begin{itemize}
        \item The User dentist has changed his password to \\
            \texttt{mY\_so0p3R\_dO0pEr\_s3cr3t\_P4s5W0rd\_1s\_rock\_candy}\\
            This could be found in the \texttt{.bash\_history} in his home
            folder.
        \item There is an empty file called ``3232candy322'' in the home directory
            Found by
            \begin{verbatim}
(find / -type f -print | grep 'candy') 
    (find / -type f -print | grep 'candy' -r search * .*)
            \end{verbatim}
    \end{itemize}
\end{exercise}

\begin{exercise}
    See dns\_sniffer.py and dnssniff.log for a sample from hellgate.
\end{exercise}

\begin{exercise}
    See collider.py. The plotting was implemented with pyplot in the script.
    For sample output see 
    \begin{itemize}
        \item Collider\_data.txt
        \item 4.png
        \item 8.png
        \item 12.png
        \item 16.png
        \item 20.png
        \item average\_time.png
        \item average\_time.png
    \end{itemize}
\end{exercise}

\begin{exercise}
    \begin{enumerate}[a)]
        \item This method would be secure if it was for sure, that Alice is the
            person who have added the second padlock.
            In this case, nobody could open the box until Bob has removed his lock
            and from this point, only Alice would be able to open the box.
        
            Now what could happen, is that not Alice is the one who adds the
            second lock, e.g. the deliveryman could add a second padlock and
            return the box to Bob.  He would never know that the padlock was not
            added by Alice and removes his own lock. Now the deliveryman would
            be able to view the content, because he owns the key to the box.
        
            For cryptographic means the circumstances are pretty much the same. You
            can not be sure, that there is no person in the middle who intercepts
            the packet and adds his own lock.
        \item  
            This method works to keep the integrity of the data. After using the
            XOR operation twice, the data is again the same as before. The order
            in which this operations are processed has no impact.
            Except for what we described in sub task a, this method would also be secure.
    \end{enumerate}
\end{exercise}

\begin{exercise}
    See hmac.py for the implementation and
    netsec2015\_exercise\_sheet\_02\_hmac.log for the HMAC of the assignment
    sheet.
\end{exercise}
\end{document}
